\documentclass[%
12pt, % Schriftgroesse
a4paper, % Papiergroesse
headsepline, % Kopfzeilenlinie
footsepline, % Fusszeilenlinie
parskip, % Abstand zw. Absaetzen
headings=normal, % etw. kleinere Ueberschriften
%draft % Testkram
]{scrartcl}
% Wir benutzen UTF8 und T1
\usepackage[utf8x]{inputenc}
\usepackage[T1]{fontenc}
% Wir schreiben deutsch
\usepackage[german]{babel}
% Mathe 
\usepackage{amsmath}
\usepackage{amsfonts}
\usepackage{amssymb}
\usepackage{amsthm}
\usepackage{xfrac}
% Referenzen
\usepackage[german]{fancyref}
% Grafiken
\usepackage{graphicx}
% Einheiten schoen gesetzt
\usepackage[locale=DE,binary-units = true]{siunitx}
\sisetup{load-configurations=binary}
% verlinkte Referenzen
\usepackage{hyperref}
% referenzen mit Namen
\usepackage{nameref}
\usepackage[automark,plainfootsepline]{scrpage2} % Kopf- und Fusszeilen
% vernünftige URL-Formatierung
\usepackage{url}
% Abstand in enumerationen
\usepackage{enumitem}
% Weniger Margins
%\usepackage[cm,myheadings]{fullpage}
\usepackage[a4paper,margin=2cm,footskip=1cm]{geometry}

% Metainformationen
\author{DLRG Ortsgruppe Langen~e.V.}
\title{Satzung}
\subtitle{}
\date{}
\subject{}
\publishers{%
im Bereich des\\
Bezirkes Cuxhaven--Osterholz e.V.\\ 
des\\
LV Niedersachsen e.V.\\
der\\
Deutschen Lebens--Rettungs--Gesellschaft e.V.
}

\renewcommand{\thesection}{\S~\arabic{section}}
\sisetup{quotient-mode = fraction,fraction-function = \sfrac}
\setlist[itemize]{noitemsep}

\pagestyle{scrheadings}

\begin{document}
\maketitle
\vspace{-0.3cm}


\section{Name, Sitz}
\label{sec:name_sitz}
\begin{enumerate}
    \item Die DLRG--Ortsgruppe Langen~e.V. der Deutschen Lebens-Rettungs-Gesellschaft e.V. ist eine Gliederung der in das Vereinsregister Hannover eingetragenen Deutschen Lebens--Rettungs--Gesellschaft Landesverband Niedersachsen e.V. und des in das Vereinsregister des Amtsgerichts Langen eingetragenen DLRG-Bezirks Cuxhaven--Osterholz e.V.
    \item Sie führt die Bezeichnung \glqq{}DLRG--Ortsgruppe Langen~e.V.\grqq{}. Sie ist in dem Vereinsregister des Amtsgerichts Langen eingetragen.
    \item Der Vereinssitz ist Langen.
    \item Die DLRG--Ortsgruppe Langen~e.V. ist Mitglied im Landessportbund.
\end{enumerate}

\section{Zweck}
\label{sec:zweck}
\begin{enumerate}
    \item Die DLRG--Ortsgruppe Langen~e.V. ist eine im Rahmen der Satzung der Deutschen Lebens--Rettungs--Gesellschaft e.V., des Landesverbandes Niedersachsen e.V. der DLRG und des DLRG--Bezirks Cuxhaven--Osterholz e.V. selbständige Organisation. Sie arbeitet grundsätzlich ehrenamtlich mit freiwilligen Helfern. Sie verfolgt ausschließlich und unmittelbar gemeinnützige Zwecke im Sinne des Abschnitts \glqq{}Steuerbegünstigte Zwecke\grqq{} der Abgabenordnung und nicht in erster Linie eigenwirtschaftliche Zwecke.\\
      Sie ist politisch, ethnisch und konfessionell neutral.
    \item Ihre Aufgabe ist auf der Grundlage sportlichen Handelns im Sinne der humanitären Tradition die Schaffung und Förderung aller Einrichtungen und Maßnahmen, die der Bekämpfung des Ertrinkungstodes dienen.
    \item Zu den Aufgaben nach Absatz 2 gehören insbesondere:\begin{itemize}
        \item Aufklärung der Bevölkerung über Gefahren am und im Wasser,
        \item Förderung des Anfängerschwimmens,
        \item Förderung des Schwimmunterrichts,
        \item Aus- und Fortbildung von Schwimmern, Rettungsschwimmern, Bootsführern, Funkern und Rettungstauchern,
        \item Aus- und Fortbildung für Hilfsmaßnahmen in Notfällen sowie die Erteilung entsprechender Befähigungszeugnisse,
        \item Planung, Organisation und Durchführung des Wasserrettungs- und Wasserbergungdienstes,
        \item Mitwirkung bei der Abwendung und Bekämpfung von Katastrophen am und im Wasser,
        \item Mitwirkung im Rahmen gesetzlicher und vertraglicher Regelungen des Rettungswachdienstes,
        \item Natur- und Umweltschutz am und im Wasser,
        \item Förderung jugendpflegerischer Arbeit.
    \end{itemize}

\end{enumerate}

\section{Mitgliedschaft}
\label{sec:mitgliedschaft}
\begin{enumerate}
    \item Ordentliche Mitglieder der DLRG-Ortsgruppe Langen~e.V. können nur natürliche Personen werden; juristische Personen, Gesellschaften, Vereinigungen und Behörden können als fördernde Mitglieder aufgenommen werden. Sie erkennen durch ihre Eintrittserklärung diese Satzung und die geltenden Ordnungen der DLRG an und übernehmen alle sich daraus ergebenden Rechte und Pflichten.
    \item Über die Aufnahme als Mitglied entscheidet der Vorstand. Ein Aufnahmeantrag gilt als angenommen, wenn er nicht bis zum Ablauf des Folgemonats abgelehnt wird.
    \item Das Mitglied wird gegenüber der übergeordneten Gliederung durch die gewählten Delegierten der DLRG--Ortsgruppe Langen~e.V. vertreten.
    \item Die Ausübung der Mitgliedsrechte ist davon abhängig, daß die Beitragszahlung für das laufende oder mindestens für das vorausgegangene Geschäftsjahr nachgewiesen ist.
    \item Das Stimmrecht kann erst nach Vollendung des 16. Lebensjahres ausgeübt werden. Wahlfunktionen können nur von Mitgliedern wahrgenommen werden, die das 18. Lebensjahr vollendet haben; ausgenommen davon sind die gewählten Vertreter der DLRG--Jugend. Das aktive und passive Wahlrecht für die DLRG--Jugend regelt die Jugendordnung.
    \item Die Mitgliedschaft endet durch Tod, Austritt, Streichung oder Ausschluß.\begin{enumerate}[noitemsep]
        \item Die Austrittserklärung eines Mitgliedes muß schriftlich einen Monat vor Ablauf des Geschäftsjahres zugegangen sein. Der Austritt wird zum Ende des Geschäftsjahres wirksam.
        \item Die Streichung als Mitglied kann bei einem Rückstand von mehr als einem Jahresbeitrag erfolgen. Auf Antrag kann die Mitgliedschaft nach Zahlung der rückständigen Beiträge fortgeführt werden.
        \item Wegen schuldhaften Verstoßes gegen die Bestimmungen dieser Satzung, der Satzung der Deutschen Lebens--Rettungs--Gesellschaft e.V., der Satzung des Landesverbandes Niedersachsen e.V. sowie der Satzung des DLRG--Bezirks Cuxhaven-Osterholz e.V. oder gegen Anordnungen aufgrund der vorgenannten Satzungen bzw. wegen unehrenhaften oder DLRG--schädigenden Verhaltens kann das zuständige Schieds- und Ehrengericht wahlweise folgende Ordnungsmaßnahmen einzeln oder gleichzeitig verhängen:\begin{itemize}
            \item Rüge, Verweis oder Verwarnung des Antragsgegners,
            \item befristeter oder dauernder Ausschluß des Antragsgegners von Wahlfunktionen in der DLRG,
            \item befristeter oder dauernde Ausschluß des Antragsgegners aus der DLRG.
        \end{itemize}
        Darüber hinaus können den Beteiligten die durch das Verfahren entstandenen Kosten ganz oder teilweise auferlegt werden. Im Übrigen regelt das Verfahren die Schieds- und Ehrengerichtsordnung.
        \item Die Mitglieder haben Jahresbeiträge zu leisten, deren Höhe von der \nameref{sec:jahreshauptversammlung} festgelegt wird. Die Mindesthöhe des Jahresbeitrages wird von der Bundestagung der DLRG festgelegt.
        \item Endet die Mitgliedschaft, ist das im Besitz befindliche DLRG--Eigentum zurückzugeben; scheidet ein Mitglied aus einer Amtstätigkeit aus, hat es die amtsbezogenen Unterlagen an die Ortsgruppe herauszugeben.
        \item Durch eigenmächtige Handlung eines Mitgliedes werden die Deutschen Lebens--Rettungs--Gesellschaft e.V. und ihre Gliederungen nicht verpflichtet.

    \end{enumerate}
\end{enumerate}

\section{Jugend}
\label{sec:jugend}
\begin{enumerate}
    \item Die DLRG--Jugend ist die Gemeinschaft der Jugendlichen in der DLRG.
    \item Die Bildung einer Jugendgruppe in der DLRG--Ortsgruppe Langen~e.V. und die damit verbundene jugendpflegerische Arbeit stellen ein besonderes Anliegen und eine bedeutende Aufgabe der DLRG dar.
    \item Inhalt und Form der Arbeit der Jugendgruppe vollziehen sich nach der Landesjugendordnung der DLRG--Jugend im Landesverband Niedersachsen e.V. sowie dem Grundsatzprogramm, die vom Landesjugendtag beschlossen werden.
\end{enumerate}


\section{Jahreshauptversammlung}
\label{sec:jahreshauptversammlung}
\begin{enumerate}
    \item Die Jahreshauptversammlung gibt die Richtlinien für die Tätigkeit der DLRG--Ortsgruppe Langen~e.V. und behandelt grundsätzliche Angelegenheiten, nimmt die Berichte des Vorstandes und der Revisoren entgegen und ist zuständig für: \begin{enumerate}[noitemsep]
        \item Wahl der Mitglieder des Vorstandes und deren Stellvertreter,
        \item Wahl der Deligierten und deren Stellvertreter zur Bezirkstagung des übergeordneten Bezirkes,
        \item Wahl des weiteren Mitgliedes der DLRG--Ortsgruppe Langen~e.V. im Bezirksrat des übergeordneten Bezirkes und dessen Stellvertreter,
        \item Wahl von zwei Revisoren und deren Stellvertreter,
        \item Bestätigung der Wahlen zum Jugendausschuß der DLRG--Ortsgruppe Langen~e.V.,
        \item Entlastung des Vorstandes,
        \item Festsetzung zeitlich begrenzter, sachbezogener Umlagen,
        \item Genehmigung des Haushaltsplanes,
        \item Beschlussfassung über ihr vorgelegte Anträge der stimmberechtigten Mitglieder nach \ref{sec:mitgliedschaft} sowie des Vorstandes der DLRG--Ortsgruppe Langen~e.V.,
        \item Festsetzung der Höhe des Jahresbeitrages,
        \item ggf. erforderliche Ergänzungswahlen.
    \end{enumerate}
    Wahlen und Bestätigungen gemäß a) bis e) werden grundsätzlich alle drei Jahre vor der Bezirkstagung des übergeordneten Bezirkes durchgeführt.
    \item Der Vorsitzende beruft die Jahreshauptversammlung ein und leitet sie.
    \item \begin{enumerate}[noitemsep]
        \item Die Jahreshauptversammlung setzt sich aus den Mitgliedern der DLRG--Ortsgruppe Langen~e.V. zusammen.
        \item Jedes stimmberechtigte Mitglied hat eine Stimme. Die Ausübung des Stimmrechts ist geregelt in \ref{sec:mitgliedschaft} Abs. 4 und 5.
      \end{enumerate}
    \item \begin{enumerate}[noitemsep]
        \item Die Jahreshauptversammlung findet einmal jährlich statt, ferner als außerordentliche Jahreshauptversammlung auf Beschluß des Vorstandes oder auf schriftlichen Antrag von mindestens 10\% der stimmberechtigten Mitglieder.
        \item Zur Jahreshauptversammlung muß die DLRG--Ortsgruppe Langen~e.V. mindestens einen Monat vorher die Mitglieder und Revisoren einladen. Die Frist beginnt mit dem auf die Absendung des Einladungsschreibens (Datum des Poststempels) folgenden Tag.
        \item Anträge zur Jahreshauptversammlung müssen mindestens zwei Wochen vorher eingegangen sein.
      \end{enumerate}
    \item Über den Inhalt der Jahreshauptversammlung ist ein Protokoll anzufertigen, vom Sitzungsleiter und Protokollführer zu unterzeichnen und auf der folgenden Jahreshauptversammlung zur Genehmigung vorzulegen.
\end{enumerate}

\section{Vorstand}
\label{sec:vorstand}
\begin{enumerate}
    \item Der Vorstand leitet die DLRG--Ortsgruppe Langen~e.V. im Rahmen dieser Satzung, der Satzung der Deutschen Lebens-Rettungs-Gesellschaft e.V., der Satzung des Landesverbandes Niedersachsen e.V. der DLRG, der Satzung des DLRG--Bezirks Cuxhaven--Osterholz e.V. sowie der Empfehlungen des Landesverbandes Niedersachsen e.V. und des übergeordneten Bezirkes. Ihm obliegt insbesondere die Ausführung der Beschlüsse der Jahreshauptversammlung sowie der Empfehlungen des übergeordneten Bezirkes und des Landesverbandes Niedersachsen e.V.
    \item Den Vorstand bilden:\begin{enumerate}[noitemsep]
        \item Vorsitzende(r),
        \item Zweiter Vorsitzende(r),
        \item Schatzmeister(in) oder Stellvertreter(in),
        \item zwei Technische Leiter,
        \item Jugendwart(in) oder ein(e) Stellvertreter(in).
      \end{enumerate}
      \setcounter{enumii}{6}
      Er kann erweitert werden höchstens um: \begin{enumerate}[noitemsep]
          \item Arzt/Ärztin oder Stellvertreter(in),
          \item Leiter(in) der Öffentlichkeitsarbeit oder Stellvertreter(in),
          \item Justitiar(in) oder Stellvertreter(in),
          \item drei Beisitzer(innen).
        \end{enumerate}
    Vorstand im Sinne des \S 26 BGB sind der Vorsitzende und zweite Vorsitzende; jeder ist allein vertretungsberechtigt. Vereinsintern ist vereinbart, daß der zweite Vorsitzende nur im nicht nachweispflichtigen Verhinderungsfalle des Vorsitzenden vertretungsberechtigt ist.
    \item Die Mitglieder des Vorstandes sowie deren Stellvertreter werden von der Jahreshauptversammlung, auf der Wahlen gemäß \ref{sec:jahreshauptversammlung} Abs. 1 anstehen, gewählt bzw. bestätigt. Die Amtszeit der Mitglieder des Vorstandes sowie deren Stellvertreter endet mit der Feststellung des Ergebnisses der jeweiligen Neuwahl bzw. mit der Abstimmung über die jeweilige Bestätigung.
    \item Schatzmeister(in) oder Stellvertreter(in) dürfen nicht zugleich Vorsitzender oder zweiter Vorsitzender sein, im Übrigen ist eine Personalunion zwischen mehreren Vorstandsämtern möglich.
    \item Die Mitglieder des Vorstandes führen ihre Ämter nach Richtlinien, die sich der Vorstand gibt.
    \item Für bestimmte Arbeitsgebiete kann der Vorstand beauftragte berufen; ihre Amtszeit endet spätestens mit der ihres zuständigen Vorstandsmitgliedes.
    \item Über den Inhalt jeder Sitzung des Vorstandes ist ein Protokoll anzufertigen, vom Sitzungsleiter zu unterzeichnen und den Vorstandsmitgliedern spätestens mit der Einladung zur nächsten Vorstandssitzung zuzuleiten.

\end{enumerate}

\section{Verhältnis zum Landesverband Niedersachsen e.V. und zum übergeordneten Bezirk}
\label{sec:verhaeltnis}
\begin{enumerate}
    \item \begin{enumerate}[noitemsep]
        \item Der Vorstand des Landesverbandes Niedersachsen e.V. der Deutschen Lebens--Rettungs--Gesellschaft ist berechtigt, die Arbeit der DLRG--Ortsgruppe Langen~e.V. zu überprüfen und in ihre sämtlichen Unterlagen Einsicht zu nehmen sowie Empfehlungen zu erteilen, die der Erfüllung der Aufgaben nach \ref{sec:zweck} dieser Satzung dienen.
        \item Der übergeordnete Bezirk hat die gleichen Rechte.
      \end{enumerate}
    \item \begin{enumerate}[noitemsep]
        \item Zu den Jahreshauptversammlungen ist der Vorstand des übergeordneten Bezirkes fristgerecht einzuladen; von allen Jahreshauptversammlungen ist dem übergeordneten Bezirkes eine Zweitschrift der Niederschrift binnen sechs Wochen zuzuleiten.
        \item Vorstandsmitglieder der Deutschen Lebens--Rettungs--Gesellschaft e.V., des Landesverbandes Niedersachsen e.V. der DLRG sowie des übergeordneten Bezirkes haben das Recht, an den Jahreshauptversammlungen sowie Zusammenkünften der Organe der DLRG--Ortsgruppe Langen~e.V. teilzunehmen; ihnen ist auf Wunsch das Wort zu erteilen.
      \end{enumerate}
    \item Nach Abschluß des Geschäftsjahres sind dem übergeordneten Bezirk zuzuleiten: \begin{enumerate}[noitemsep]
        \item Technischer Bericht,
        \item Beitragsrechnung,
        \item Jahresabschluß nebst angeordneten Unterlagen,
        \item aus sämtlichen fälligen Zahlungsverpflichtungen gegenüber dem übergeordneten Bezirk zu zahlende Beiträge,
        \item Nachweis der Erledigung von Auflagen, deren Befolgung von Organen des Landesverbandes Niedersachsen e.V. der DLRG oder des übergeordneten Bezirkes verlangt worden ist.
      \end{enumerate}
    \item DieTermine, zu denen Unterlagen vorzulegen und Zahlungen zu leisten sind, werden durch die Organe des übergeordneten Bezirkes festgesetzt.
    \item Werden die Verpflichtungen aus dem Absatz 3 unvollständig oder nicht termingerecht erfüllt, ist den Mitgliedern und Delegierten der DLRG--Ortsgruppe Langen~e.V. im nächsten Rat bzw. in der nächsten Tagung des übergeordneten Bezirkes vom Fälligkeitstermin ab das Stimmrecht versagt.
\end{enumerate}

\section{Ordnungsbestimmungen}
\label{sec:ordnungsbestimmungen}
\begin{enumerate}
    \item Das Geschäftsjahr entspricht dem Kalenderjahr.
    \item Verwaltungskosten dürfen nur insoweit erstattet werden, als sie dem Satzungszweck (\ref{sec:zweck}) entsprechen. Vergütungen dürfen nur insoweit gewahrt werden, wie sie mit der Gemeinnützigkeit vereinbar sind, Mittel des Vereins dürfen nur für satzungsgemäße Zwecke verwendet werden. Die DLRG Mitglieder erhalten keine Zuwendungen aus Mitteln des Vereins.
    \item \begin{enumerate}[noitemsep]
        \item Einladungen und Anträge zu Zusammenkünften der Organe müssen stets schriftlich erfolgen. Einladungen müssen außerdem die vorgesehene Tagesordnung enthalten. Das Einladungsschreiben gilt dem Mitglied als zugegangen, wenn es an die letzte von ihm dem Verein schriftlich bekanntgegebene Adresse gerichtet ist. Bei Familien, Ehepaaren und Paaren genügt \emph{eine} schriftliche Einladung.
        \item Einladungen zur Jahreshauptversammlung müssen schriftlich oder durch einmalige Veröffentlichung in der für offizielle Bekanntmachungen bestimmten Tageszeitung, jeweils unter Angabe der gesamten Tagesordnung, erfolgen. Dasselbe gilt für alle weiteren Veröffentlichungen. Wenn die DLRG--Ortsgruppe Langen~e.V. ein eigenes Vereinsorgan herausgibt (\ref{sec:vereinsorgan}), so können Einladungen zur Jahreshauptversammlungen darin erfolgen.
        \item Fristgerecht eingereichte Anträge müssen den zur Zusammenkunft eingeladenen Teilnehmern spätestens bei Beginn der Zusammenkunft vorgelegt werden.
      \end{enumerate}
    \item \begin{enumerate}[noitemsep]
        \item Die Jahreshauptversammlung ist ohne Rücksicht auf die Zahl der anwesende Stimmberechtigten beschlußfahig; zur Beschlußfähigkeit des Vorstandes ist die Anwesenheit von mehr als die Hälfte der Stimmberechtigten erforderlich. 
        \item Besteht keine Beschlußfähigkeit des Vorstandes, kann innerhalb von vier Wochen eine neue Zusammenkunft durchgeführt werden, die ohne Rücksicht auf die Zahl der anwesenden Stimmberechtigten beschlußfähig ist. Zu Ihr muß mindestens zwei Wochen vorher schriftlich unter Bekanntgabe der Tagesordnung eingeladen werden.
      \end{enumerate}
    \item \begin{enumerate}[noitemsep]
        \item Gewählt wird grundsätzlich geheim; wenn kein Stimmberechtigter widerspricht, kann offen gewählt werden. Gewählt ist, wer die Mehrheit der abgegebenen Stimmen auf sich vereinigt.
        \item Sonstige Beschlüsse der Jahreshauptversammlung und des Vorstandes werden, soweit diese Satzung nichts anderes vorschreibt, mit einfacher Mehrheit der abgegebenen Stimmen gefaßt. Bei Stimmengleichheit gilt ein Antrag als abgelehnt.\\
          Abstimmungen erfolgen offen, soweit nicht geheime Abstimmung beschlossen wird.
      \end{enumerate}
    \item Einem Organ vorgelegte Dringlichkeitsanträge können nur behandelt werden, wenn \num{2/3} der anwesenden Stimmberechtigten die Behandlung zulassen. Satzungsänderung und Wahlen können kein Gegenstand von Dringlichkeitsanträge sein.
    \item \begin{enumerate}[noitemsep]
        \item Abstimmungen führt grundsätzlich der Leiter der Zusammenkunft durch.
        \item Für Wahlen wird grundsätzlich ein Wahlausschuß gebildet; er kann vom anwesenden Vertreter des übergeordneten Bezirkes oder des Landesverbandes Niedersachsen e.V. der DLRG geleitet werden.
      \end{enumerate}
    \item Wer in der Deutschen Lebens--Rettungs--Gesellschaft e.V. oder in einer ihrer Gliederung haupt- oder nebenamtlich tätig ist, kann keine Wahlfunktion im Vorstand der DLRG--Ortsgruppe Langen~e.V. wahrnehmen.
    \item Bei Streitigkeiten innerhalb der DLRG ist vor Einleitung gerichtlicher Schritte das zuständige Schieds- und Ehrengericht anzurufen.
\end{enumerate}

\section{Ordnungen der DLRG}
\label{sec:ordnungen}
\begin{enumerate}
    \item Im Rahmen der Ausbildungs- und Lehrtätigkeit nimmt die DLRG Prüfungen ab. Art, Inhalt und Durchführung werden durch die Prüfungsordnung der DLRG und deren Ausführungsbestimmungen geregelt; sie sind für den Prüfer und Prüfungsteilnehmer bindend.
    \item Zur Durchführung von Jahreshauptversammlungen und Vorstandssitzungen gilt die Geschäftsordnung der DLRG.
    \item Die Finanz- und Materialwirtschaft sowie die Rechnungslegung regelt die Wirtschaftsordnung der DLRG.
    \item Das Verfahren vor dem Schieds- und Ehrengericht regelt die Schieds- und Ehrengerichtsordnung der DLRG, die vom Präsidialrat beschlossen wird.
    \item Das Verfahren für Ehrungen regelt die Ehrungsordnung der DLRG.
    \item Soweit für den Landesverband Niedersachsen e.V. der DLRG Ergänzungen der vorgenannten Ordnungen beschlossen wurden, gelten diese für die DLRG--Ortsgruppe Langen~e.V.
\end{enumerate}

\section{Warenzeichen und Material}
\label{sec:warenzeichen}
\begin{enumerate}
    \item Die Buchstabenfolge DLRG sowie die Verbandzeichen sind im Warenzeichenregister des Deutschen Patentamtes München warenzeichenrechtlich geschützt.
    \item Die Verwendung der Buchstabenfolge und der Verbandszeichen wird durch eine Gestaltungsordnung (Standards) geregelt; sie wird vom Präsidialrat der DLRG erlassen.
    \item Das zur Erfüllung ihrer Aufgaben benötigte Material (DLRG-Material) wird von der DLRG vertrieben.
    \item Die DLRG Ortsgruppe Langen~e.V. ist verpflichtet, dafür Sorge zu tragen, daß das zur Aufgabenerfüllung verwendete Material, das nicht von der Materialstelle der DLRG bezogen wird, der Gestaltungsordnung entspricht und zur Erfüllung der in \ref{sec:zweck} dieser Satzung aufgeführten Aufgaben geeignet ist.
\end{enumerate}

\section{Vereinsorgan}
\label{sec:vereinsorgan}
Die DLRG--Ortsgruppe Langen~e.V. kann ein offizielles Vereinsorgan herausgeben.

\section{Satzungsänderungen}
\label{sec:satzungsaenderungen}
\begin{enumerate}
    \item Satzungsänderung können nur von einer Jahreshauptversammlung beschlossen werden. Zu einem satzungsändernden Beschluß ist eine Mehrheit von \num{2/3} erforderlich. Eine Satzungsänderung bedarf der Zustimmung des Vorstandes des Landesverbandes Niedersachsen e.V. der DLRG.
    \item Die beantragte Satzungsänderung muß im Wortlaut und mit schriftlicher Begründung mit der Einladung zur Jahreshauptversammlung bekanntgegeben werden.
    \item Der Vorstand wird ermächtigt, Satzungsänderung, die vom zuständigen Registergericht oder Finanzamt für erforderlich gehalten werden, selbst mit einfacher Mehrheit zu beschließen und beim Registergericht anzumelden. Dasselbe gilt für Satzungsänderungen, die vom Vorstand des Landesverbandes Niedersachsen e.V. der DLRG aus verbandsinternen Gründen für erforderlich gehalten werden.
\end{enumerate}

\section{Auflösung}
\label{sec:aufloesung}
\begin{enumerate}
    \item Die Auflösung der DLRG Ortsgruppe Langen~e.V. kann nur in einer zu diesem Zweck mindestens sechs Wochen vorher einberufenen außerordentlichen Jahreshauptversammlung mit einer Mehrheit von \num{3/4} der anwesenden Stimmberechtigten beschlossen werden.
    \item Bei Auflösung der DLRG Ortsgruppe Langen~e.V. oder bei Wegfall ihres bisherigen Zwecks fällt ihr Vermögen an den Landesverband Niedersachsen e.V. der DLRG, das er ausschließlich im Bereich des Landes Niedersachsen für seine gemeinnützigen sportlichen Zweck verwenden darf, bzw. an den übergeordneten Bezirk.
\end{enumerate}

\section{Inkrafttreten der Satzung}
\label{sec:inkrafttreten}
\begin{enumerate}
    \item Die Satzung bedarf zu ihrer Wirksamkeit der Zustimmung des Vorstandes des Landesverbandes Niedersachsen e.V. der DLRG.
    \item Die Satzung ist am 28.01.2002 auf der Jahreshauptversammlung der DLRG Ortsgruppe Langen~e.V. beschlossen und am \hspace{1.5cm} unter der Nr. 616 in das Vereinsregister des Amtsgerichts Langen eingetragen worden.
\end{enumerate}



\end{document}