\documentclass[%
12pt, % Schriftgroesse
a4paper, % Papiergroesse
headsepline, % Kopfzeilenlinie
footsepline, % Fusszeilenlinie
parskip, % Abstand zw. Absaetzen
headings=normal, % etw. kleinere Ueberschriften
%draft % Testkram
]{scrartcl}
% Wir benutzen UTF8 und T1
\usepackage[utf8x]{inputenc}
\usepackage[T1]{fontenc}
% Wir schreiben deutsch
\usepackage[ngerman]{babel}
% Mathe 
\usepackage{amsmath}
\usepackage{amsfonts}
\usepackage{amssymb}
\usepackage{amsthm}
\usepackage{xfrac}
% Referenzen
\usepackage[german]{fancyref}
% Grafiken
\usepackage{graphicx}
% Einheiten schoen gesetzt
\usepackage[locale=DE,binary-units = true]{siunitx}
\sisetup{load-configurations=binary}
% verlinkte Referenzen
\usepackage{hyperref}
% referenzen mit Namen
\usepackage{nameref}
\usepackage[automark,plainfootsepline]{scrpage2} % Kopf- und Fusszeilen
% vernünftige URL-Formatierung
\usepackage{url}
% Abstand in enumerationen
\usepackage{enumitem}
% Weniger Margins
%\usepackage[cm,myheadings]{fullpage}
\usepackage[a4paper,margin=2cm,footskip=1cm]{geometry}

% Metainformationen
\author{DLRG Ortsgruppe Langen~e.V.}
\title{Satzung}
\subtitle{}
\date{}
\subject{}
\publishers{%
im Bereich des\\
Bezirkes Cuxhaven--Osterholz e.V.\\ 
des\\
LV Niedersachsen e.V.\\
der\\
Deutschen Lebens--Rettungs--Gesellschaft e.V.
}

\renewcommand{\thesection}{\S~\arabic{section}}
\sisetup{quotient-mode = fraction,fraction-function = \sfrac}
\setlist[itemize]{noitemsep}

\pagestyle{scrheadings}

\begin{document}
\maketitle
\vspace{-0.3cm}

\addsec{Präambel}
Die DLRG bildet durch ihre Mitglieder und Gliederungen die größte freiwillige und führende Wasserrettungsorganisation Deutschlands und der Welt. In ihr finden alle Mitglieder und Gliederungen eine ehrenamtlich und humanitär wirkende Gesellschaft zur Verhinderung von Ertrinkungsfällen vor. Alle Gliederungen, die den Namen der DLRG führen, erkennen den bindenden Charakter dieser Gesellschaft an und verpflichten sich, ihr ganzes Tun und Handeln im Sinne dieser bundesweiten Gesellschaft auszurichten. Gegenseitiges Vertrauen, Glaubwürdigkeit, gemeinschaftliches Handeln sowie die Übereinstimmung von Wort und Tat bilden die Grundlage des verbandlichen Umgangs. Sie begründen die menschliche Qualität der Mitglieder und die Stärke der DLRG.

\section{Name, Sitz}
\label{sec:name_sitz}
\begin{enumerate}
    \item Die DLRG--Ortsgruppe Langen~e.V. der Deutschen Lebens-Rettungs-Gesellschaft e.V. ist eine Gliederung der in das Vereinsregister des Amtsgerichts Hannover eingetragenen Deutschen Lebens--Rettungs--Gesellschaft Landesverband Niedersachsen e.V. und des in das Vereinsregister des Amtsgerichts Tostedt eingetragenen DLRG--Bezirks Cuxhaven--Osterholz e.V.
    \item Sie führt die Bezeichnung \glqq{}DLRG--Ortsgruppe Langen~e.V.\grqq{}. Sie ist in dem Vereinsregister des Amtsgerichts Tostedt eingetragen.
    \item Vereinssitz ist Geestland.
    \item Geschäftsjahr ist das Kalenderjahr.
    \item Die DLRG--Ortsgruppe Langen~e.V. ist Mitglied im Landessportbund.
\end{enumerate}

\section{Zweck}
\label{sec:zweck}
\begin{enumerate}
    \item Die vordringliche Aufgabe der DLRG--Ortsgruppe Langen~e.V. ist auf der Grundlage sportlichen Handelns im Sinne der humanitären Tradition die Schaffung und Förderung aller Einrichtungen und Maßnahmen, die der Bekämpfung des Ertrinkungstodes dienen (Förderung der Rettung aus Lebensgefahr).
    \item Zu den Kernaufgaben nach Absatz 1 gehören insbesondere:\begin{itemize}
        \item Frühzeitige und fortgesetzte Information über Gefahren im und am Wasser sowie über sicherheitsbewusstes Verhalten,
        \item Ausbildung im Schwimmen und in der Selbstrettung,
        \item Ausbildung im Rettungsschwimmen,
        \item Weiterqualifizierung von Rettungsschwimmern für Ausbildung und Einsatz,
        \item Organisation und Durchführung eines flächendeckenden Wasserrettungsdienstes im Rahmen und als Teil der allgemeinen Gefahrenabwehr des Landes, der Landkreise und Gemeinden.
      \end{itemize}
    \item Eine weitere bedeutende Aufgabe der DLRG ist die Jugendarbeit und die Nachwuchsförderung.
    \item Zu den Aufgaben gehören auch die \begin{enumerate}[noitemsep]
        \item Aus- und Fortbildung in Erster Hilfe und im Sanitätswesen
        \item Unterstützung und Gestaltung freizeitbezogener Maßnahmen am, im und auf dem Wasser,
        \item Durchführung rettungssportlicher Übungen und Wettkämpfe,
        \item Aus- und Fortbildung ehrenamtlicher Mitarbeiter, insbesondere auch in den Bereichen Führung, Organisation und Verwaltung.
      \end{enumerate}

\end{enumerate}

\section{Gemeinnützigkeit}
\label{sec:gemeinnuetzigkeit}
\begin{enumerate}
    \item Die DLRG-Ortsgruppe Langen~e.V. ist eine im Rahmen der Satzung der Deutschen Lebens--Rettungs--Gesellschaft~e.V., des Landesverbands Niedersachsen e.V. der DLRG und des DLRG--Bezirks Cuxhaven--Osterholz~e.V. selbständige Organisation. Sie arbeitet grundsätzlich ehrenamtlich mit freiwilligen Helfern. Sie verfolgt ausschließlich und unmittelbar gemeinnützige Zwecke im Sinne des Abschnitts \glqq{}Steuerbegünstigte Zwecke\grqq{} der Abgabenordnung. Sie ist selbstlos tätig und verfolgt nicht in erster Linie eigenwirtschaftliche Zwecke. Sie ist politisch, ethnisch und konfessionell neutral.
    \item Mittel der DLRG dürfen nur für satzungsgemäße Zwecke verwendet werden. Die Mitglieder erhalten in ihrer Eigenschaft als Mitglieder keine Zuwendungen aus Mitteln der DLRG. Diese darf niemanden Ausgaben erstatten, die ihrem Zweck fremd sind, oder unverhältnismäßig hohe Vergütungen gewähren.
    \item Verwaltungskosten dürfen nur insoweit erstattet werden, als sie dem Satzungszweck (\ref{sec:zweck}) entsprechen. Vergütungen dürfen nur gewährt werden, wie sie mit der Gemeinnützigkeit vereinbar sind.
    \item Für Dienstleistungen, die die DLRG--Ortsgruppe Langen e.V. im Rahmen des Satzungszwecks (\ref{sec:zweck}, Abs. 2 und 3) erbringt, kann von Dritten ein Entgelt verlangt werden; dessen Höhe sich nach der Gebührenordnung richtet, die der Landesverbandsrat erlässt.
\end{enumerate}

\section{Mitgliedschaft}
\label{sec:mitgliedschaft}
\begin{enumerate}
    \item Mitglieder der DLRG-Ortsgruppe Langen~e.V. können natürliche und juristische Personen des Privatrechtes und des öffentlichen Rechtes werden. Sie erkennen durch ihre schriftliche Eintrittserklärung diese Satzung und die geltenden Ordnungen der DLRG an und übernehmen alle sich daraus ergebenden Rechte und Pflichten.
    \item Über die Aufnahme als Mitglied entscheidet der Vorstand. Ein Aufnahmeantrag gilt als angenommen, wenn er nicht bis zum Ablauf des Folgemonats abgelehnt wird.
    \item Das Mitglied wird gegenüber der übergeordneten Gliederung durch die gewählten Delegierten der DLRG--Ortsgruppe Langen~e.V. vertreten.
    \item Die Ausübung der Mitgliedsrechte ist davon abhängig, dass die Beitragszahlung für das laufende oder mindestens für das vorausgegangene Geschäftsjahr nachgewiesen ist.
    \item Das Stimmrecht kann erst nach Vollendung des 16. Lebensjahres ausgeübt werden. Wahlfunktionen können nur von Mitgliedern wahrgenommen werden, die das 18. Lebensjahr vollendet haben; ausgenommen davon sind die gewählten Vertreter der DLRG--Jugend. Das aktive und passive Wahlrecht für die DLRG--Jugend regelt die Jugendordnung.
    \item Die Mitgliedschaft endet durch Tod, Austritt, Streichung oder Ausschluss.\begin{enumerate}[noitemsep]
        \item Die Austrittserklärung eines Mitgliedes muss schriftlich einen Monat vor Ablauf des Geschäftsjahres zugegangen sein. Der Austritt wird zum Ende des Geschäftsjahres wirksam.
        \item Die Streichung als Mitglied kann erfolgen ab einem Rückstand mit einem Jahresbeitrag, wenn der Rückstand mindestens einmal unter Fristsetzung erfolglos angemahnt wurde. Auf Antrag kann die Mitgliedschaft nach Zahlung der rückständigen Beiträge fortgeführt werden.
        \item Wegen schuldhaften Verstoßes gegen die Bestimmungen dieser Satzung, der Satzung der Deutschen Lebens--Rettungs--Gesellschaft~e.V., der Satzung des Landesverbandes Niedersachsen~e.V. sowie der Satzung des DLRG--Bezirks Cuxhaven-Osterholz~e.V. oder gegen Anordnungen aufgrund dieser Satzung bzw. wegen unehrenhaften oder DLRG--schädigenden Verhaltens kann das zuständige Schieds- und Ehrengericht wahlweise folgende Ordnungsmaßnahmen einzeln oder gleichzeitig verhängen:\begin{enumerate}
            \item Rüge,
            \item Verweis,
            \item zeitlicher oder dauernder Ausschluss von Ämtern,
            \item zeitliche oder dauernde Aberkennung des passiven Wahlrechts,
            \item Aberkennung ausgesprochener Ehrungen,
            \item zeitliches oder dauerndes Verbot des Zutritts zu bestimmten oder allen Einrichtungen und Veranstaltungen, ausgenommen Zusammenkünfte der Organe,
            \item Ausschluss.
        \end{enumerate}
        Darüber hinaus können den Beteiligten die durch das Verfahren entstandenen Kosten ganz oder teilweise auferlegt werden. Im Übrigen regelt das Verfahren die Schieds- und Ehrengerichtsordnung.
        \item Die Mitglieder haben Jahresbeiträge zu leisten, deren Höhe von der \nameref{sec:jahreshauptversammlung} festgelegt wird. 
        \item Endet die Mitgliedschaft, ist das im Besitz befindliche DLRG--Eigentum zurückzugeben, scheidet ein Mitglied aus einer Amtstätigkeit aus, hat es die amtsbezogenen Unterlagen an die Ortsgruppe herauszugeben.
        \item Durch eigenmächtige Handlung eines Mitgliedes werden die Deutschen Lebens--Rettungs--Gesellschaft e.V. und ihre Gliederungen nicht verpflichtet.

    \end{enumerate}
\end{enumerate}

\section{Jugend}
\label{sec:jugend}
\begin{enumerate}
    \item Die DLRG--Jugend ist die Gemeinschaft der Jugendlichen in der DLRG.
    \item Die Bildung einer Jugendgruppe in der DLRG--Ortsgruppe Langen~e.V. und die damit verbundene jugendpflegerische Arbeit stellen ein besonderes Anliegen und eine bedeutende Aufgabe der DLRG dar. Die freiwillige selbständige Übernahme und Ausführung von Aufgaben der Jugendhilfe erfolgen auf der Grundlage der gemeinnützigen Zielsetzung der DLRG.
    \item Inhalt und Form der Arbeit der Jugendgruppe vollziehen sich nach der Landesjugendordnung der DLRG--Jugend im Landesverband Niedersachsen~e.V. sowie dem Grundsatzprogramm, die vom Landesjugendtag beschlossen werden.
    \item Der Vorstand wird im Jugendvorstand durch eines seiner Mitglieder vertreten.
\end{enumerate}


\section{Jahreshauptversammlung}
\label{sec:jahreshauptversammlung}
\begin{enumerate}
    \item Die Jahreshauptversammlung gibt die Richtlinien für die Tätigkeit der DLRG--Ortsgruppe Langen~e.V. und behandelt grundsätzliche Angelegenheiten, nimmt die Berichte des Vorstandes und der Revisoren entgegen und ist zuständig für: \begin{enumerate}[noitemsep]
        \item Wahl der Mitglieder des Vorstandes und deren Stellvertreter gem. \ref{sec:vorstand},
        \item Wahl von zwei Revisoren und deren Stellvertreter,
        \item Wahl der Delegierten und deren Stellvertreter zur Bezirkstagung des übergeordneten Bezirkes,
        \item Wahl des weiteren Mitgliedes der DLRG--Ortsgruppe Langen~e.V. im Bezirksrat des übergeordneten Bezirkes und dessen Stellvertreter,
        \item Entlastung des Vorstandes,
        \item Festsetzung zeitlich begrenzter, sachbezogener Umlagen,
        \item Genehmigung des Haushaltsplanes,
        \item Beschlussfassung über ihr vorgelegte Anträge der stimmberechtigten Mitglieder nach \ref{sec:mitgliedschaft} sowie des Vorstandes der DLRG--Ortsgruppe Langen~e.V.,
        \item Festsetzung der Höhe des Jahresbeitrages,
        \item Satzungsänderungen,
        \item ggf. erforderliche Ergänzungswahlen.
    \end{enumerate}
    Wahlen gemäß a) bis d) werden grundsätzlich alle drei Jahre vor der Bezirkstagung des übergeordneten Bezirkes durchgeführt.
    \item Der Vorsitzende beruft die Jahreshauptversammlung ein und leitet sie.
    \item \begin{enumerate}[noitemsep]
        \item Die Jahreshauptversammlung setzt sich aus den Mitgliedern der DLRG--Ortsgruppe Langen~e.V. zusammen.
        \item Jedes stimmberechtigte Mitglied hat eine Stimme. Die Ausübung des Stimmrechts ist geregelt in \ref{sec:mitgliedschaft} Abs. 4 und 5.
      \end{enumerate}
    \item \begin{enumerate}[noitemsep]
        \item Die Jahreshauptversammlung findet jährlich einmal statt, ferner als außerordentliche Jahreshauptversammlung auf Beschluss des Vorstandes oder auf schriftlichen Antrag von mindestens 10\% der stimmberechtigten Mitglieder.
        \item Zur Jahreshauptversammlung muss die DLRG--Ortsgruppe Langen~e.V. mindestens einen Monat vorher die Mitglieder und Revisoren einladen. Die Frist beginnt mit dem auf die Absendung des Einladungsschreibens (Datum des Poststempels) folgenden Tag. Die Einladung erfolgt in Textform.
        \item Anträge zur Jahreshauptversammlung müssen mindestens zwei Wochen vorher in Textform eingegangen sein.
      \end{enumerate}
    \item Über den Inhalt jeder Jahreshauptversammlung ist ein Protokoll anzufertigen, vom Sitzungsleiter und Protokollführer zu unterzeichnen und auf der folgenden Jahreshauptversammlung zur Genehmigung vorzulegen. % hier sollte vielleicht noch nen Punkt rein, dass eine Zugänglichmachung auch reicht
\end{enumerate}

\section{Vorstand}
\label{sec:vorstand}
\begin{enumerate}
    \item Der Vorstand leitet die DLRG--Ortsgruppe Langen~e.V. im Rahmen dieser Satzung, der Satzung der Deutschen Lebens-Rettungs-Gesellschaft~e.V., der Satzung des Landesverbandes Niedersachsen e.V. der DLRG, der Satzung des DLRG--Bezirks Cuxhaven--Osterholz~e.V. sowie der Empfehlungen des Landesverbandes Niedersachsen~e.V. und des übergeordneten Bezirkes. Ihm obliegt insbesondere die Ausführung der Beschlüsse der Jahreshauptversammlung sowie der Empfehlungen des übergeordneten Bezirkes und des Landesverbandes Niedersachsen~e.V.
    \item Den Vorstand bilden:\begin{enumerate}[noitemsep]
        \item Vorsitzende(r),
        \item Zweite(r) Vorsitzende(r),
        \item Schatzmeister(in) oder Stellvertreter(in),
        \item Leiter(in) Ausbildung oder Stellvertreter(in),
        \item Leiter(in) Einsatz oder Stellvertreter(in),
        \item Vorsitzende(r) der DLRG-Jugend oder ein(e) Stellvertreter(in).
      \end{enumerate}
      \setcounter{enumii}{7}
      Er kann erweitert werden höchstens um: \begin{enumerate}[noitemsep]
          \item Arzt/Ärztin oder Stellvertreter(in),
          \item Leiter(in) der Öffentlichkeitsarbeit oder Stellvertreter(in),
          \item Justitiar(in) oder Stellvertreter(in),
          \item drei Beisitzer(innen).
        \end{enumerate}
    \item Vorstand im Sinne des \S 26 BGB sind der Vorsitzende und zweite Vorsitzende; jeder ist allein vertretungsberechtigt. Vereinsintern ist vereinbart, dass der zweite Vorsitzende nur im nicht nachweispflichtigen Verhinderungsfalle des Vorsitzenden vertretungsberechtigt ist.
    \item Die Mitglieder des Vorstandes sowie deren Stellvertreter werden von der Jahreshauptversammlung, auf der Wahlen gemäß \ref{sec:jahreshauptversammlung} Abs. 1 anstehen, gewählt. Die Amtszeit der Mitglieder des Vorstandes sowie deren Stellvertreter endet mit der Feststellung des Ergebnisses der jeweiligen Neuwahl.
    \item Eine Personalunion zwischen mehreren Vorstandsämtern ist möglich. Ausgeschlossen ist eine Personalunion zwischen dem Vorstand gem. \S 26 BGB (OG--Satzung \ref{sec:vorstand}, Abs. 3) und dem Schatzmeister oder Stellvertreter.
    \item Die Mitglieder des Vorstandes führen ihre Ämter nach Richtlinien, die sich der Vorstand gibt.
    \item Für bestimmte Arbeitsgebiete kann der Vorstand Beauftragte berufen; ihre Amtszeit endet spätestens mit der ihres zuständigen Vorstandsmitgliedes.
    \item Über den Inhalt jeder Sitzung des Vorstandes ist ein Protokoll anzufertigen, vom Sitzungsleiter und Protokollführer zu unterzeichnen und den Vorstandsmitgliedern spätestens mit der Einladung zur nächsten Vorstandssitzung zuzuleiten.

\end{enumerate}

\section{Verhältnis zum Landesverband Niedersachsen e.V. und zum übergeordneten Bezirk}
\label{sec:verhaeltnis}
\begin{enumerate}
    \item \begin{enumerate}[noitemsep]
        \item Der Vorstand des Landesverbandes Niedersachsen~e.V. der Deutschen Lebens--Rettungs--Gesellschaft sowie der übergeordnete Bezirk sind berechtigt, die Arbeit der DLRG--Ortsgruppe Langen~e.V. zu überprüfen und in ihre sämtlichen Unterlagen Einsicht zu nehmen sowie Empfehlungen zu erteilen, die der Erfüllung der Aufgaben nach \ref{sec:zweck} dieser Satzung dienen.
      \end{enumerate}
    \item \begin{enumerate}[noitemsep]
        \item Zu den Jahreshauptversammlungen ist der Vorstand des übergeordneten Bezirkes fristgerecht einzuladen; von allen Jahreshauptversammlungen ist dem Vorstand des übergeordneten Bezirkes eine Zweitschrift der Niederschrift binnen sechs Wochen zuzuleiten.
        \item Vorstandsmitglieder der Deutschen Lebens--Rettungs--Gesellschaft~e.V., des Landesverbandes Niedersachsen~e.V. der DLRG sowie des übergeordneten Bezirkes haben das Recht, an den Jahreshauptversammlungen sowie Zusammenkünften der Organe der DLRG--Ortsgruppe Langen~e.V. teilzunehmen; ihnen ist auf Wunsch das Wort zu erteilen.
      \end{enumerate}
    \item Nach Abschluss des Geschäftsjahres sind dem übergeordneten Bezirk zuzuleiten: \begin{enumerate}[noitemsep]
        \item Technischer Bericht,
        \item Beitragsrechnung,
        \item Jahresabschluss nebst angeordneten Unterlagen,
        \item aus sämtlichen fälligen Zahlungsverpflichtungen gegenüber dem übergeordneten Bezirk zu zahlende Beiträge,
        \item Nachweis der Erledigung von Auflagen, deren Befolgung von Organen des Landesverbandes Niedersachsen~e.V. der DLRG oder des übergeordneten Bezirkes verlangt worden ist.
      \end{enumerate}
    \item DieTermine, zu denen Unterlagen vorzulegen und Zahlungen zu leisten sind, werden durch die Organe des übergeordneten Bezirkes festgesetzt.
    \item Werden die Verpflichtungen aus dem Absatz 3 unvollständig oder nicht termingerecht erfüllt, ist den Mitgliedern und Delegierten der DLRG--Ortsgruppe Langen~e.V. im nächsten Rat bzw. in der nächsten Tagung des übergeordneten Bezirkes vom Fälligkeitstermin ab das Stimmrecht versagt.
\end{enumerate}

\section{Ordnungsbestimmungen}
\label{sec:ordnungsbestimmungen}
\begin{enumerate}
    \item \begin{enumerate}[noitemsep]
        \item Einladungen und Anträge zu Zusammenkünften der Organe müssen stets in Textform erfolgen. Einladungen müssen außerdem die vorgesehene Tagesordnung enthalten. Das Einladungsschreiben gilt dem Mitglied als zugegangen, wenn es an die letzte von ihm dem Verein in Textform bekanntgegebene Adresse gerichtet ist. Bei Familien, Ehepaaren und nichtehelichen Lebensgemeinschaften genügt eine Einladung in Textform.
        \item Wenn die DLRG--Ortsgruppe Langen~e.V. ein eigenes Vereinsorgan herausgibt (\ref{sec:vereinsorgan}), so können Einladungen und Anträge zur Jahreshauptversammlung darin erfolgen.
        \item Zu Beginn der Versammlung sind die der Versammlung vorzulegenden Anträge an die stimmberechtigt anwesenden Mitglieder auszuhändigen.
      \end{enumerate}
    \item \begin{enumerate}[noitemsep]
        \item Die Jahreshauptversammlung ist ohne Rücksicht auf die Zahl der anwesende Stimmberechtigten beschlussfähig; zur Beschlussfähigkeit des Vorstandes ist die Anwesenheit von mehr als die Hälfte der Stimmberechtigten erforderlich. 
        \item Besteht keine Beschlussfähigkeit des Vorstandes, kann innerhalb von vier Wochen eine neue Zusammenkunft durchgeführt werden, die ohne Rücksicht auf die Zahl der anwesenden Stimmberechtigten beschlussfähig ist. Zu Ihr muss mindestens zwei Wochen vorher in Textform unter Bekanntgabe der Tagesordnung eingeladen werden.
      \end{enumerate}
    \item \begin{enumerate}[noitemsep]
        \item Gewählt wird grundsätzlich geheim; wenn kein Stimmberechtigter widerspricht, kann offen gewählt werden. Gewählt ist, wer die Mehrheit der abgegebenen Stimmen auf sich vereinigt. Enthaltungen werden mitgezählt.
        \item Wahlen können als Blockwahl durchgeführt werden, wenn kein Stimmberechtigter widerspricht.
        \item Sonstige Beschlüsse der Jahreshauptversammlung und des Vorstandes werden, soweit diese Satzung nichts anderes vorschreibt, mit einfacher Mehrheit der abgegebenen Stimmen gefasst. Stimmenthaltungen und ungültige Stimmen werden nicht mitgezählt. Bei Stimmengleichheit gilt ein Antrag als abgelehnt.\\
          Abstimmungen erfolgen offen, soweit nicht geheime Abstimmung beschlossen wird.
      \end{enumerate}
    \item Einem Organ vorgelegte Dringlichkeitsanträge können nur behandelt werden, wenn \num{2/3} der anwesenden Stimmberechtigten die Behandlung zulassen. Satzungsänderung und Wahlen können kein Gegenstand von Dringlichkeitsanträgen sein.
    \item \begin{enumerate}[noitemsep]
        \item Abstimmungen führt grundsätzlich der Leiter der Zusammenkunft durch.
        \item Für Wahlen wird grundsätzlich ein Wahlausschuss gebildet; er kann vom anwesenden Vertreter des übergeordneten Bezirkes oder des Landesverbandes Niedersachsen~e.V. der DLRG geleitet werden.
      \end{enumerate}
    \item Bei Streitigkeiten innerhalb der DLRG ist vor Einleitung gerichtlicher Schritte das zuständige Schieds- und Ehrengericht anzurufen.
\end{enumerate}

\section{Ordnungen der DLRG}
\label{sec:ordnungen}
\begin{enumerate}
    \item Im Rahmen der Ausbildungs- und Lehrtätigkeit nimmt die DLRG Prüfungen ab. Art, Inhalt und Durchführung werden durch die Prüfungsordnung der DLRG und deren Ausführungsbestimmungen geregelt; sie sind für Prüfer und Prüfungsteilnehmer bindend.
    \item Zur Durchführung von Jahreshauptversammlungen und Vorstandssitzungen gilt die Geschäftsordnung der DLRG.
    \item Die Finanz- und Materialwirtschaft sowie die Rechnungslegung regelt die Wirtschaftsordnung der DLRG.
    \item Das Verfahren vor dem Schieds- und Ehrengericht regelt die Schieds- und Ehrengerichtsordnung der DLRG.
    \item Das Verfahren für Ehrungen regelt die Ehrungsordnung der DLRG.
    \item Soweit für den Landesverband Niedersachsen~e.V. der DLRG Ergänzungen der vorgenannten Ordnungen beschlossen wurden, gelten diese für die DLRG--Ortsgruppe Langen~e.V.
\end{enumerate}

\section{Material}
\label{sec:material}
\begin{enumerate}
    \item Das zur Erfüllung ihrer Aufgaben benötigte Material (DLRG--Material) wird von der DLRG vertrieben und soll von der Materialstelle der DLRG auf dem Dienstwege bezogen werden.
    \item Die DLRG Ortsgruppe Langen~e.V. ist verpflichtet, dafür Sorge zu tragen, dass das zur Aufgabenerfüllung verwendete Material, das nicht von der Materialstelle der DLRG bezogen wird, der Gestaltungsordnung entspricht und zur Erfüllung der in \ref{sec:zweck} dieser Satzung aufgeführten Aufgaben geeignet ist.
\end{enumerate}

\section{Vereinsorgan}
\label{sec:vereinsorgan}
Die DLRG--Ortsgruppe Langen~e.V. kann ein offizielles Vereinsorgan herausgeben.

\section{Satzungsänderungen}
\label{sec:satzungsaenderungen}
\begin{enumerate}
    \item Satzungsänderung können nur von der Jahreshauptversammlung beschlossen werden. Zu einem satzungsändernden Beschluss ist eine Mehrheit von \num{2/3} der anwesenden Stimmberechtigten erforderlich.\\
      Eine Satzungsänderung bedarf der Zustimmung des Vorstandes des Landesverbandes Niedersachsen~e.V. der DLRG.
    \item Die beantragte Satzungsänderung muss im Wortlaut und mit schriftlicher Begründung mit der Einladung zur Jahreshauptversammlung bekanntgegeben werden.
    \item Der Vorstand wird ermächtigt, Satzungsänderungen, die vom zuständigen Registergericht oder Finanzamt für erforderlich gehalten werden, selbst mit einfacher Mehrheit zu beschließen und beim Registergericht anzumelden. Dasselbe gilt für Satzungsänderungen, die vom Vorstand des Landesverbandes Niedersachsen~e.V. der DLRG aus verbandsinternen Gründen für erforderlich gehalten werden.
\end{enumerate}

\section{Auflösung}
\label{sec:aufloesung}
\begin{enumerate}
    \item Die Auflösung der DLRG Ortsgruppe Langen~e.V. kann nur in einer zu diesem Zweck mindestens sechs Wochen vorher einberufenen außerordentlichen Jahreshauptversammlung mit einer Mehrheit von \num{3/4} der anwesenden Stimmberechtigten beschlossen werden.
    \item Bei Auflösung der DLRG Ortsgruppe Langen~e.V. oder bei Wegfall steuerbegünstigter Zwecke fällt ihr Vermögen an den übergeordneten Bezirk. Für den Fall, dass der Bezirk das Vermögen nicht übernimmt, fällt dieses an den Landesverband e.V. der DLRG, der es unmittelbar und ausschließlich für gemeinnützige Zwecke zu verwenden hat.
\end{enumerate}

\section{Inkrafttreten der Satzung}
\label{sec:inkrafttreten}
\begin{enumerate}
    \item Die Satzung bedarf zu ihrer Wirksamkeit der Zustimmung des Vorstandes des Landesverbandes Niedersachsen e.V. der DLRG.
    \item Die Satzung ist am 28.01.2002 auf der Jahreshauptversammlung der DLRG Ortsgruppe Langen~e.V. beschlossen und am \hspace{1.5cm} unter der Nr. 616 in das Vereinsregister des Amtsgerichts Langen eingetragen worden.\\
      Die Satzung ist durch Beschluss der Jahreshauptversammlung vom xx.yy.xxxx geändert in der Präambel sowie den \S\S 1 (\nameref{sec:name_sitz}), 2 (\nameref{sec:zweck}), 3 (\nameref{sec:gemeinnuetzigkeit}), 4 (\nameref{sec:mitgliedschaft}), 5 (\nameref{sec:jugend}), 6 (\nameref{sec:jahreshauptversammlung}), 7 (\nameref{sec:vorstand}), 8 (\nameref{sec:verhaeltnis}), 9 (\nameref{sec:ordnungsbestimmungen}), 10 (\nameref{sec:ordnungen}), 11 (\nameref{sec:material}), 13 (\nameref{sec:satzungsaenderungen}) und 14 (\nameref{sec:aufloesung}). Die Änderung wurde im Vereinsregister unter der Nr. VR~110322 des Amtsgerichts Tostedt eingetragen.
\end{enumerate}

\end{document}